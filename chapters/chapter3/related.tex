\section{Related Work}
\label{chap3-sec:related}
% \smallskip
\noindent{\textbf{Non-private Subgraph Counting.}}~~Subgraph counting has been extensively studied in a non-private setting (see \cite{Ribeiro_CS21} for a recent survey). 
% Differentially priate graph analysis has been widely studied in the literature. 
% In particular, triangle counting is considered one of the most basic task 
Examples of subgraphs include triangles \cite{Bera_PODS20,Eden_FOCS15,Kolountzakis_IM12,Wu_TKDE16},  4-cycles \cite{Bera_STACS17,Kallaugher_PODS19,Manjunath_ESA11,McGregor_PODS20}, $k$-stars \cite{Aliakbarpour_Alg18,Gonen_DM11}, and 
$k$-hop paths \cite{Bjorklund_ICALP19,Kartun-Giles_TWC18}. 

Here, the main challenge is to reduce the computational time of counting these subgraphs in large-scale graph data. 
One of the simplest approaches is edge sampling \cite{Bera_PODS20,Eden_FOCS15,Wu_TKDE16}, which randomly 
% and independently 
samples edges in a graph. 
Edge sampling outperforms other sampling methods 
% approaches 
(e.g., node sampling, triangle sampling) \cite{Wu_TKDE16} and is also adopted in \cite{Imola_USENIX22} 
% to improve communication efficiency in 
for private triangle counting. 

Although our triangle algorithm also samples user-pairs, ours is different from edge sampling in two ways. 
First, our algorithm does not sample an edge but samples a pair of users who may or may not be a friend. 
% have an edge between them. 
Second, our algorithm samples user-pairs that share no common users to avoid the increase of the privacy budget $\epsilon$ as well as to reduce the time complexity 
(see Section~\ref{chap3-sec:triangle} for details). 

\smallskip
\noindent{\textbf{Private Subgraph Counting.}}~~Differentially private subgraph counting has been widely studied, and the previous work assumes either the central \cite{Ding_TKDE21,Karwa_PVLDB11,Kasiviswanathan_TCC13} or local \cite{Imola_USENIX21,Imola_USENIX22,Sun_CCS19,Ye_ICDE20,Ye_TKDE21} models. 
The central model assumes a centralized social network and has a data breach issue, as explained in Section~\ref{chap3-sec:intro}. 

Subgraph counting in the local model has recently attracted attention. 
Sun \textit{et al.} \cite{Sun_CCS19} propose subgraph counting algorithms 
% under the assumption 
assuming 
that each user knows all friends' friends. 
However, this assumption does not hold in many social networks; e.g., Facebook users can change their settings so that anyone cannot see their friend lists. 
Therefore, we make a minimal assumption -- each user knows only her friends. 

% Some studies \cite{Imola_USENIX21,Imola_USENIX22,Ye_ICDE20,Ye_TKDE21} focus on this setting. 

In this setting, recent studies propose triangle \cite{Imola_USENIX21,Imola_USENIX22,Ye_ICDE20,Ye_TKDE21} and $k$-star \cite{Imola_USENIX21} counting algorithms. 
% As described in Section~\ref{chap3-sec:intro}, 
For $k$-stars, Imola \textit{et al.} \cite{Imola_USENIX21} propose a one-round algorithm that is order optimal and show that it provides a very small estimation error. 
% Triangles are much more challenging than $k$-stars. 
% Ye \textit{et al.} \cite{Ye_ICDE20} propose a one-round algorithm that applies the RR to each bit of the neighbor list and then counts the number of triangles in the noisy graph. 
% They also propose an algorithm to reduce the bias in the estimate \cite{Ye_TKDE21} (though there is no proof that the estimate is unbiased). 
% Imola \textit{et al.} \cite{Imola_USENIX21} 
For triangles, they
propose a one-round algorithm that applies the RR to 
% bits for smaller user IDs in 
each bit of the neighbor list 
and then calculates an unbiased estimate of triangles from the noisy graph. 
We call this algorithm \AlgRRTri{}. 
Imola \textit{et al.} \cite{Imola_USENIX22} show that \AlgRRTri{} provides a much smaller estimation error than the one-round triangle algorithms in \cite{Ye_ICDE20,Ye_TKDE21}. 
In \cite{Imola_USENIX22}, they also reduce the time complexity of \AlgRRTri{} 
% from $O(n^3)$ to $O(n^2)$ 
by using the ARR (Asymmetric RR), which samples each 1 (edge) after applying the RR. 
We call this algorithm \AlgARRTri{}. 
In this paper, we use \AlgRRTri{} and \AlgARRTri{} as baselines in triangle counting. 
% compare our triangle algorithms with \AlgRRTri{} and \AlgARRTri{} in our theoretical analysis and with \AlgARRTri{} in our experiments (we cannot apply \AlgRRTri{} to large-scale graphs, as it is too inefficient; see ). 
For 4-cycles, there is no existing algorithm under LDP, to our knowledge. 
Thus, we compare our shuffle algorithm with its local version, which does not shuffle the obfuscated data. 

For triangles, Imola \textit{et al.} also propose a two-round local algorithm in \cite{Imola_USENIX21} and significantly reduce its download cost in \cite{Imola_USENIX22}. 
Although we focus on one-round algorithms, 
% as described in Section~\ref{chap3-sec:intro}. 
we show in 
% Section~\ref{chap3-sec:experiments} 
\conference{the full version \cite{Imola_CCSFull22}}\arxiv{Appendix~\ref{chap3-sec:two-round}} 
that our one-round algorithm is comparable to the two-round algorithm in \cite{Imola_USENIX22}, which requires a lot of user effort and synchronization, in terms of accuracy. 

% In this paper, we compare our triangle algorithms with \AlgRRTri{} and \AlgARRTri{} in our theoretical analysis and with \AlgARRTri{} in our experiments (we cannot apply \AlgRRTri{} to large-scale graphs, as it is too inefficient; see ). 

% one-round local triangle counting \cite{Imola_USENIX21,Imola_USENIX22,Ye_ICDE20,Ye_TKDE21}, \AlgARRTri{} \cite{Imola_USENIX22}, \AlgRRTri{} \cite{Imola_USENIX21}. Explain  and \AlgARRTri{} in detail because they are compared with our algorithms.

\smallskip
\noindent{\textbf{Shuffle Model.}}~~The privacy amplification by shuffling has been recently studied in \cite{Balle_CRYPTO19,Cheu_EUROCRYPT19,Erlingsson_SODA19,Feldman_FOCS21}. 
Among them, the privacy amplification bound by Feldman \textit{et al.} \cite{Feldman_FOCS21} is the state-of-the-art -- it provides a smaller $\epsilon$ than other bounds, such as \cite{Balle_CRYPTO19,Cheu_EUROCRYPT19,Erlingsson_SODA19}. 
%and is more general than the bound in \cite{Cheu_EUROCRYPT19} that is specific to binary RR. 
Girgis \textit{et al.} \cite{Girgis_CCS21} 
consider multiple interactions between users and the data collector and 
show a better bound than the bound in \cite{Feldman_FOCS21} when used with composition. 
However, the bound in \cite{Feldman_FOCS21} outperforms the bound in \cite{Girgis_CCS21} when used without composition. 
% (which is the case with this work that considers a single interaction). 
Because our work focuses on a single interaction and does not use the composition, 
% Therefore, 
we use the bound in \cite{Feldman_FOCS21}. 

The shuffle model has been applied to tabular data \cite{Meehan_ICLR22,Wang_PVLDB20} and gradients \cite{Girgis_AISTATS21,Liu_AAAI21} in federated learning. 
Meehan \textit{et al.} \cite{Meehan_ICLR22} construct a graph from public auxiliary information and determine a permutation of obfuscated data using the graph to reduce re-identification risks. 
Liew \textit{et al.} \cite{Liew_SIGMOD22} propose network shuffling, which shuffles obfuscated data via random walks on a graph. 
%, where users exchange their obfuscated data 
%each user relays her obfuscated data to her neighbor
Note that both \cite{Meehan_ICLR22} and \cite{Liew_SIGMOD22} use graph data to shuffle another type of data. 
% neither \cite{Meehan_ICLR22} or \cite{Liew_SIGMOD22} shuffles graph data itself (they shuffle another data using graph data). 
To our knowledge, our work is the first to shuffle graph data itself. 
