%-------------------------------------------------------------------------------
\section{Related Work}
\label{sec:related}
%-------------------------------------------------------------------------------
% Here we review the prior work related to ours. 
% Here we review the previous work on graph DP (Differential Privacy), LDP (Local DP), and upper/lower-bounds.

\smallskip
\noindent{\textbf{Graph DP.}}~~DP on graphs has been widely studied, with most prior work being in the centralized model \cite{Blocki_FOCS12,Chen_PoPETs20,Day_SIGMOD16,Hay_ICDM09,Karwa_PVLDB11,Kasiviswanathan_TCC13,Nissim_STOC07,Raskhodnikova_arXiv15,Raskhodnikova_Encyclopedia16,Song_arXiv18,Wang_PAKDD13,Wang_TDP13}. 
% Some studies proposed algorithms for counting subgraphs in this model. 
% For example, Karwa \textit{et al.} \cite{Karwa_PVLDB11} proposed an algorithm for counting triangles and $k$-stars by adding the Cauchy noise to the true count. 
% Day \textit{et al.} \cite{Day_SIGMOD16} ...
% Kasiviswanathan \textit{et al.} \cite{Kasiviswanathan_TCC13} 
In this model, a number of algorithms have been proposed for releasing subgraph counts \cite{Karwa_PVLDB11,Kasiviswanathan_TCC13,Song_arXiv18}, degree distributions \cite{Day_SIGMOD16,Hay_ICDM09,Raskhodnikova_arXiv15}, eigenvalues and eigenvectors 
\cite{Wang_PAKDD13}, and synthetic graphs \cite{Chen_PoPETs20,Wang_TDP13}. 

% Algorithms for graph DP in the local model are much fewer, and \cite{Qin_CCS17,Ye_ICDE20,Zhang_USENIX20} are such examples. 
% Some studies proposed 
There has also been a handful of work on graph algorithms in the local DP model~\cite{Qin_CCS17,Sun_CCS19,Ye_ICDE20,Ye_TKDE21,Zhang_USENIX20}. 
For example, Qin \textit{et al.} \cite{Qin_CCS17} propose an algorithm for generating synthetic graphs. 
% while
% Ye \textit{et al.} \cite{Ye_ICDE20} 
% provide a method
% for graph metric estimation under LDP. 
% that perturbs 
% a neighbor list 
% an adjacency matrix by the RR (Randomized Response) 
% and each user's degree by the Laplacian noise. 
Zhang \textit{et al.} \cite{Zhang_USENIX20} propose an algorithm for software usage analysis under LDP, where a node represents a software component (e.g., function in a code) and an edge represents a control-flow between components. 
% None of these works focus on subgraph counts. 
Neither of these works focus on subgraph counts. 

Sun \textit{et al.} \cite{Sun_CCS19} propose an algorithm for counting subgraphs in the local model under the assumption that each user allows her friends to see all her connections. 
However, this assumption does not hold in many practical scenarios; e.g., a Facebook user can change her setting so that friends cannot see her connections. Therefore, we assume that each user knows only her friends rather than all of her friends' friends. 
The algorithms in \cite{Sun_CCS19} cannot be applied to this setting.

Ye \textit{et al.} \cite{Ye_ICDE20} propose a one-round algorithm for estimating graph metrics including the clustering coefficient. 
Here they apply Warner's RR (Randomized Response) to an adjacency matrix. 
However, it introduces a very large bias for triangle counts.
In \cite{Ye_TKDE21}, they propose a method for reducing the bias in the estimate of triangle counts. 
However, the method in \cite{Ye_TKDE21} introduces some approximation, and it is unclear whether their estimate is unbiased. 
In this paper, we propose a one-round algorithm for triangles that uses empirical estimation as a post-processing step, and prove that our estimate is unbiased. 
We also show in Appendix~\ref{sec:RR_emp} that our one-round algorithm significantly outperforms the one-round algorithm in \cite{Ye_ICDE20}. 
Moreover, we show in Section~\ref{sec:experiments} that our two-rounds algorithm significantly outperforms our one-round algorithm.

Our work also differs from \cite{Sun_CCS19,Ye_ICDE20,Ye_TKDE21} in that we provide lower-bounds on the estimation error.

% Our work differs from these in two ways: (i) our work provides algorithms and theoretical performance guarantees for subgraph counts, 
% (ii) we also 
% \colorB{provide} 
% lower-bounds on the estimation error. 
% We note that although Warner's RR (Randomized Response) has been applied to an adjacency matrix in \cite{Qin_CCS17,Ye_ICDE20} for different purposes than ours, it can be used for counting triangles. 
% However, it suffers from a very large estimation error. 
% Our one-round algorithm for triangles uses empirical estimation as a post-processing step, and we show in Appendix~\ref{sec:RR_emp} that this empirical estimation step significantly reduces the estimation error.

\smallskip
\noindent{\textbf{LDP.}}~~Apart from graphs, a number of works have looked at analyzing statistics (e.g., discrete distribution estimation\cite{Acharya_AISTATS19,Fanti_PoPETs16,Kairouz_ICML16,Kairouz_JMLR16,Murakami_USENIX19,Wang_USENIX17,Ye_ISIT17}, 
% and 
heavy hitters \cite{Bassily_STOC15,Bassily_NIPS17,Qin_CCS16}) under LDP. 
% Warner's RR \cite{Warner_JASA65}, $k$-ary RR \cite{Kairouz_ICML16}, RAPPOR \cite{Erlingsson_CCS14}, ...

However, they use LDP in the context of tabular data, and do not consider the kind of complex interdependency in graph data (as described in Section~\ref{sec:intro}). 
% it can happen that their algorithms do not work well for graph data which have complex interdependency. 
For example, the RR with empirical estimation is optimal in the low privacy regimes for estimating a distribution for tabular data \cite{Kairouz_ICML16,Kairouz_JMLR16}. 
We apply the RR and empirical estimation to counting triangles, and show that it is suboptimal and significantly outperformed by a more sophisticated 
two-rounds 
algorithm. 
% with more interaction between users and a data collector. 
% This example shows that directly applying existing LDP techniques does not work well. 

\smallskip
\noindent{\textbf{Upper/lower-bounds.}}~~Finally, we note that existing work on upper-bounds and lower-bounds cannot be directly applied to our setting. 
For example, there are upper-bounds
\cite{Acharya_AISTATS19,Kairouz_ICML16,Kairouz_JMLR16,Ye_ISIT17,Joseph_ArXiv19,Joseph_ArXiv19_Gauss} and
lower-bounds
\cite{Acharya_arXiv20,Duchi_ArXiv14,Duchi_ArXiv17,Joseph_ArXiv19,Duchi_ArXiv19,
Joseph_ArXiv19_Gauss, Joseph_SODA20} on the estimation error (or sample complexity) in 
% estimating a discrete distribution on personal data. 
distribution estimation of tabular data. 
% can be found in \cite{Acharya_AISTATS19,Duchi_ArXiv14,Kairouz_ICML16,Kairouz_JMLR16,Ye_ISIT17}. 
However, they assume that each original data value is independently sampled from an underlying distribution. 
They cannot be directly applied to our graph setting, because each triangle and each $k$-star involve multiple edges and are not independent (as described in Section~\ref{sec:intro}). 
% Acharya \textit{et al.} \cite{Acharya_arXiv20} provides lower-bounds on ...
Rashtchian \textit{et al.} \cite{Rashtchian_arXiv20} provide lower-bounds on communication complexity (i.e., number of queries) of vector-matrix-vector queries for estimating subgraph counts. 
However, their lower-bounds are not on the estimation error, and cannot be applied to our problem.
