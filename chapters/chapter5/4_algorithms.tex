In this section, we design private algorithms for hierarchical clustering which work on any input graph. In Section~\ref{sec:hc-poly}, we propose a polynomial time $(\alpha, O(\frac{n^{2.5}}{\epsilon}))$ approximation algorithm, where $\alpha$ is the best approximation ratio of a black-box, \emph{non-private} hierarchical clustering algorithm. Then, in Section~\ref{sec:hc-exp}, we show that the exponential mechanism is a $(1, O(\frac{n^{2} \log n}{\epsilon}))$-approximation algorithm, implying our lower bound is tight. The proofs of the results in this section appear in Appendix~\ref{app:hc-cuts}

\subsection{Polynomial-Time Algorithm}~\label{sec:hc-poly}

Our algorithm makes use of a recent algorithm which releases a sanitized, synthetic graph $G'$ that approximates the cuts in the private graph $G$~\citep{eliavs2020differentially,arora2019differentially}. Via post-processing, it is then possible to run a non-private, black-box clustering algorithm. We are able to relate the cost in $G'$ to that of $G$ by reducing the cost $\cost_G(T)$ to a sum of cuts. We start by defining the notion of $G'$ approximating the cuts in $G$.
\begin{defn}\label{def:cut-approx}
    For a given graph $G = (V, E, w)$, we say $G' = (V, E', w')$ is an $(\alpha_n, \beta_n)$-approximation to cut queries in $G$ if for all $S \subseteq V$, we have
    \begin{multline*}
        (1-\alpha_n) w(S, \overline{S}) - \beta_n \min \{|S|, n - |S|\} \\ \leq w'(S, \overline{S}) \leq (1+\alpha_n) w(S, \overline{S}) + \beta_n \min \{|S|, n - |S|\}.
    \end{multline*}
\end{defn}

As we alluded, earlier work shows that it is possible to release an $(\tilde{O}(\frac{1}{\epsilon\sqrt{n}}), \tilde{O}(\frac{\sqrt{n}}{\epsilon}))$-approximation to cut queries while satisfying differential privacy. Using this result, we are able to run any blackbox hierarchical clustering algorithm, and by post-processing, the final clustering $T'$ will still satisfy privacy. Even though $T'$ is computed only viewing $G'$, we are able to relate $\cost_{G}(T')$ to $\cost_G^*$ using the fact that $G'$ approximates the cuts in $G$, and a decomposition of $\cost_{G'}(T')$ into a sum of cuts. This idea recently appeared in~\citet{agarwal2022sublinear}, and is a critical component of our theorem. In the end, we obtain the following:
\begin{thm}
Given an $(a_n, 0)$-approximation to the cost objective of hierarchical clustering, 
there exists an $(\epsilon, \delta)$-DP algorithm which, with probability at least $0.8$, is a  $((1+o(1))a_n, O(n^{2.5} \frac{ \log^2 n \log^2 \frac{1}{\delta}}{\epsilon}))$-approximation algorithm to the cost objective.
\end{thm}

Plugging in a state-of-the-art, $\sqrt{\log n}$ hierarchical clustering algorithm of~\citet{charikar2017approximate}, we obtain a $((1+o(1)) \sqrt{\log n}, \tilde{O}(\frac{n^{2.5}}{\epsilon}))$-approximation. In a graph with total edge weight $W$, we have $W \leq \cost_G(T) \leq nW$, and thus an approximation is possible if $W > \frac{n^{1.5}}{\epsilon}$. This means the graph can have an average degree of $\frac{\sqrt{n}}{\epsilon}$.

\subsection{Exponential Mechanism}~\label{sec:hc-exp}
We consider an algorithm based on the well-known exponential mechanism~\citep{mcsherry2007mechanism}. This algorithm takes exponential time, but achieves greater performance that is nearly tight with our lower bound (showing that the lower bound can't be improved significantly from an information-theoretic point of view).

The exponential mechanism $M : \calX \rightarrow \calY$ releases an element from $\calY$ with probability proportional to 
\[
    \Pr[M(X) = Y] \propto e^{\epsilon u_X(Y) / (2S)},
\]
where $u_X(Y)$ is a utility function, and $S = \max_{X,X',Y}|u_X(Y) - u_{X'}(Y)|$ is the sensitivity of the utility function in $X$. This ubiquitous mechanism satisfies $(\epsilon, 0)$-DP. 

In our setting, we use the utility function $u_G(T) = -\cost_G(T)$.
The sensitivity is bounded in the following fact.
\begin{fact}
For two adjacent input graphs $G = (V,E,w)$ and $G' = (V,E,w')$, we have for all trees $T$ that $|\cost_{G}(T) - \cost_{G'}(T)| \leq n$.
\end{fact}
\noindent\textit{Proof:}
    We can write the difference as as 
    \begin{align*}
        &|\cost_G(T) - \cost_{G'}(T)| \\ 
        &= \textstyle{\left|\sum_{u,v \in V^2} (w(u,v) - w'(u,v))|\texttt{leaves}(T[u\wedge v])| \right|} \\
        &\leq \textstyle{\sum_{u,v \in V^2} |w(u,v) - w'(u,v))| \cdot |\texttt{leaves}(T[u\wedge v])|} \\
        &\leq n \textstyle{\sum_{u,v \in V^2} |w(u,v) - w'(u,v)| \leq n}. \hfill \qed
    \end{align*}
Having controlled the sensitivity, we can apply utility results for the exponential mechanism.
\begin{lem}\label{lem:exp-util}
There exists an $(\epsilon, 0)$-DP, $(1, O(\frac{n^2 \log n}{\epsilon}))$-approximation algorithm for hierarchical clustering.
\end{lem}

Thus, the exponential mechanism improves on the cost, and shows that private hierarchical clustering can be done on graphs with average degree $O(\frac{n}{\epsilon})$.